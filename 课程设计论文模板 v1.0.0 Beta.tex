\documentclass[12pt,oneside,a4paper,fleqn]{ctexart}
\usepackage[headsep=2cm,headheight=0.5cm,footskip=0cm]{geometry}
\usepackage{caption}
\usepackage{graphicx}
\usepackage{setspace}
\usepackage{titletoc}
\usepackage{titlesec}
\usepackage{ctex}
\usepackage{fancyhdr}
\geometry{left=2.1cm,right=1.8cm,top=2.6cm,bottom=1.3cm} 
% 页边距设置,先用geometry包再用fancyhdr包


\pagestyle{fancy}
\fancyfoot{}
\fancyfoot[R]{\thepage}
\fancyhead{}
\fancyhead[c]{\includegraphics[width=\textwidth]{figure/2.png}}
\renewcommand{\headrulewidth}{0.4pt}
\renewcommand{\footrulewidth}{0.4pt}
%设置页眉和页脚


\usepackage{background}
\usepackage{dashrule}	%虚线包
%因为没找到直接的竖直虚线,所以采用了水平虚线,然后旋转90度的方法
\SetBgScale{1}
\SetBgAngle{90}	%旋转了90度
\SetBgColor{black}
\SetBgContents{\hdashrule[-2cm][c]{\textheight}{0.5pt}{1.5mm}}
\SetBgHshift{0cm}		%背景的水平位移
\SetBgVshift{10cm}
%设置装订线


\renewcommand{\abstractname}{\textbf{
    {\zihao{2}题目:$\times \times \times $标题}\\
    \vspace*{1.5em}
\zihao{4}摘\quad 要 \vspace*{0.5em}}}
%自定义摘要


\renewcommand{\contentsname}{\zihao{3} \heiti{目\quad 录} \vspace{0.5em}}
%自定义目录


%%%%%%%%%%%%%%%%%%%%%%%%自定义目录样式%%%%%%%%%%%%%%%%%%%%%%%%%%%
\titlecontents{section}[1cm]{\bf \zihao{-4}}{\contentslabel{2.5em}}{}{\titlerule*[0.5pc]{$\cdots$}\contentspage\hspace*{1cm}}
\titlecontents{subsection}[2cm]{\zihao{-4}}{\contentslabel{2.5em}}{}{\titlerule*[0.5pc]{$\cdots$}\contentspage\hspace*{1cm}}
\titlecontents{subsubsection}[3cm]{\zihao{-4}}{\contentslabel{2.5em}}{}{\titlerule*[0.5pc]{$\cdots$}\contentspage\hspace*{1cm}}
%%%%%%%%%%%%%%%%%%%%%%%%%%参考文献%%%%%%%%%%%%%%%%%%%%%%%%
\usepackage{cite}


%%%%%%%%%%%%%%%%%%%%%%%%%%%准备结束%%%%%%%%%%%%%%%%%%%%%%%%%%%%%%

\begin{document}
%%%%%%%%%%%%%%%%%%%%%%%%%%%标题页开始%%%%%%%%%%%%%%%%%%%%%%%%%%%
    \begin{titlepage}
        
        \heiti
        \ 
        \vspace{3em}
        \begin{center}
            \includegraphics{figure/1}\newline
            {\zihao{0}课程设计(论文)}\\
            \vspace{15em}
            \noindent\makebox[70pt][c]{\zihao{-3}课程名称:}\ \ \underline{\makebox[10em]{\zihao{4}$\times \times \times $}}\\
            \vspace{1em}
            \makebox[70pt][s]{\zihao{-3}题目:}\ \ \underline{\makebox[10em]{\zihao{4} $\times \times \times $}}\\
            \vspace{1em}
            \makebox[70pt][s]{\zihao{-3}院(系):}\ \ \underline{\makebox[10em]{\zihao{4} $\times \times \times $工程学院}}\\
            \vspace{1em}
            \makebox[70pt][s]{\zihao{-3}专业班级:}\ \ \underline{\makebox[10em]{\zihao{4}$\times \times \times $}}\\
            \vspace{1em}
            \makebox[70pt][s]{\zihao{-3}姓名:}\ \ \underline{\makebox[10em]{\zihao{4}$\times \times \times $}}\\
            \vspace{1em}
            \makebox[70pt][s]{\zihao{-3}学号:}\ \ \underline{\makebox[10em]{\zihao{4} $\times \times \times $}}\\
            \vspace{1em}
            \makebox[70pt][s]{\zihao{-3}指导教师:}\ \ \underline{\makebox[10em]{\zihao{4}$\times \times \times $}}\\
            \vspace{10em}
            {\zihao{4}2022年1月5号}
        \end{center}
    \end{titlepage}
%%%%%%%%%%%%%%%%%%%%%%%%%标题页结束%%%%%%%%%%%%%%%%%%%%%%%%%%%%%
\newpage
%%%%%%%%%%%%%%%%%%%%%%%%%%摘要页开始%%%%%%%%%%%%%%%%%%%%%%%%%%%%

\begin{abstract}
    %\thispagestyle{empty}
    \addcontentsline{toc}{section}{摘要}
    \songti
    \zihao{-4}
    \begin{spacing}{1.25}
        摘要又称概要、内容提要,意思是摘录要点或摘录下来的要点。 [1]  摘要是以提供文献内容梗概为目的,不加评论和补充解释,简明、确切地记述文献重要内容的短文。其基本要素包括研究目的、方法、结果和结论。具体地讲就是研究工作的主要对象和范围,采用的手段和方法,得出的结果和重要的结论,有时也包括具有情报价值的其它重要的信息。
        \newline
        \newline
        \textbf{关键字:}{$\times \times \times $}
    \end{spacing}
    \pagenumbering{Roman}
\end{abstract}

%%%%%%%%%%%%%%%%%%%%%%%%摘要页结束%%%%%%%%%%%%%%%%%%%%%%%%%%%%%%
\newpage
%%%%%%%%%%%%%%%%%%%%%%%%目录页开始%%%%%%%%%%%%%%%%%%%%%%%%%%%%%%
\tableofcontents
\pagenumbering{arabic}




\thispagestyle{empty}
%%%%%%%%%%%%%%%%%%%%%%%%目录页结束%%%%%%%%%%%%%%%%%%%%%%%%%%%%%%
\newpage
%%%%%%%%%%%%%%%%%%%%%%%%%正文开始%%%%%%%%%%%%%%%%%%%%%%%%%%%%%%
\setcounter{page}{1}

\vspace{0.5em} \section{\heiti $\times \times \times $} \vspace{0.5em}\begin{spacing}{1.25}


摘要又称概要、内容提要,意思是摘录要点或摘录下来的要点。 [1]  摘要是以提供文献内容梗概为目的,不加评论和补充解释,简明、确切地记述文献重要内容的短文。其基本要素包括研究目的、方法、结果和结论。具体地讲就是研究工作的主要对象和范围,采用的手段和方法,得出的结果和重要的结论,有时也包括具有情报价值的其它重要的信息。


\end{spacing}\vspace{0.5em} \subsection{\heiti $\times \times \times $}\begin{spacing}{1.25}

摘要又称概要、内容提要,意思是摘录要点或摘录下来的要点。 [1]  摘要是以提供文献内容梗概为目的,不加评论和补充解释,简明、确切地记述文献重要内容的短文。其基本要素包括研究目的、方法、结果和结论。具体地讲就是研究工作的主要对象和范围,采用的手段和方法,得出的结果和重要的结论,有时也包括具有情报价值的其它重要的信息。


\end{spacing}\vspace{0.5em} \subsection{\heiti $\times \times \times $}\begin{spacing}{1.25}

摘要又称概要、内容提要,意思是摘录要点或摘录下来的要点。 [1]  摘要是以提供文献内容梗概为目的,不加评论和补充解释,简明、确切地记述文献重要内容的短文。其基本要素包括研究目的、方法、结果和结论。具体地讲就是研究工作的主要对象和范围,采用的手段和方法,得出的结果和重要的结论,有时也包括具有情报价值的其它重要的信息。


\end{spacing}\vspace{0.5em} \subsection{\heiti $\times \times \times $}\begin{spacing}{1.25}

摘要又称概要、内容提要,意思是摘录要点或摘录下来的要点。 [1]  摘要是以提供文献内容梗概为目的,不加评论和补充解释,简明、确切地记述文献重要内容的短文。其基本要素包括研究目的、方法、结果和结论。具体地讲就是研究工作的主要对象和范围,采用的手段和方法,得出的结果和重要的结论,有时也包括具有情报价值的其它重要的信息。

\end{spacing}\vspace{0.5em} \section{\heiti $\times \times \times $} \vspace{0.5em}\begin{spacing}{1.25}

摘要又称概要、内容提要,意思是摘录要点或摘录下来的要点。 [1]  摘要是以提供文献内容梗概为目的,不加评论和补充解释,简明、确切地记述文献重要内容的短文。其基本要素包括研究目的、方法、结果和结论。具体地讲就是研究工作的主要对象和范围,采用的手段和方法,得出的结果和重要的结论,有时也包括具有情报价值的其它重要的信息。

\end{spacing}\vspace{0.5em} \subsection{\heiti $\times \times \times $}\begin{spacing}{1.25}

摘要又称概要、内容提要,意思是摘录要点或摘录下来的要点。 [1]  摘要是以提供文献内容梗概为目的,不加评论和补充解释,简明、确切地记述文献重要内容的短文。其基本要素包括研究目的、方法、结果和结论。具体地讲就是研究工作的主要对象和范围,采用的手段和方法,得出的结果和重要的结论,有时也包括具有情报价值的其它重要的信息。



\end{spacing}\vspace{0.5em} \subsection{\heiti $\times \times \times $}\begin{spacing}{1.25}

摘要又称概要、内容提要,意思是摘录要点或摘录下来的要点。 [1]  摘要是以提供文献内容梗概为目的,不加评论和补充解释,简明、确切地记述文献重要内容的短文。其基本要素包括研究目的、方法、结果和结论。具体地讲就是研究工作的主要对象和范围,采用的手段和方法,得出的结果和重要的结论,有时也包括具有情报价值的其它重要的信息。

\end{spacing}\vspace{0.5em} \subsection{\heiti $\times \times \times $}\begin{spacing}{1.25}

摘要又称概要、内容提要,意思是摘录要点或摘录下来的要点。 [1]  摘要是以提供文献内容梗概为目的,不加评论和补充解释,简明、确切地记述文献重要内容的短文。其基本要素包括研究目的、方法、结果和结论。具体地讲就是研究工作的主要对象和范围,采用的手段和方法,得出的结果和重要的结论,有时也包括具有情报价值的其它重要的信息。

\end{spacing}\vspace{0.5em} \section*{\heiti 参考文献} \vspace{0.5em}\begin{spacing}{1.25}\addcontentsline{toc}{section}{参考文献}
    \begin{thebibliography}{99}  

        \bibitem{ref1}郭莉莉,白国君,尹泽成,魏惠芳. “互联网+”背景下沈阳智慧交通系统发展对策建议[A]. 中共沈阳市委、沈阳市人民政府.第十七届沈阳科学学术年会论文集[C].中共沈阳市委、沈阳市人民政府:沈阳市科学技术协会,2020:4.
        \bibitem{ref2}陈香敏,魏伟,吴莹. “文化+人工智能”视阈下文化创意产业融合发展实践及路径研究[A]. 中共沈阳市委、沈阳市人民政府.第十七届沈阳科学学术年会论文集[C].中共沈阳市委、沈阳市人民政府:沈阳市科学技术协会,2020:4.
        \bibitem{ref3}田晓曦,刘振鹏,彭宝权. 地方高校开展教育人工智能深度融合的路径探究[A]. 中共沈阳市委、沈阳市人民政府.第十七届沈阳科学学术年会论文集[C].中共沈阳市委、沈阳市人民政府:沈阳市科学技术协会,2020:5.
        \bibitem{ref4}柏卓君,潘勇,李仲余.彩色多普勒超声在早期胚胎停育诊断中的应用[J].影像研究与医学应用,2020,4(18):129-131.
        \bibitem{ref5}杨芸.我院2018年人血白蛋白临床应用调查与分析[J].上海医药,2020,41(17):34-35+74.
        \bibitem{ref6}高小峰,徐涛,胡昱,刘春风,乔雨,周欣竹.现场浇筑白鹤滩大坝混凝土强度性能试验研究[J].浙江工业大学学报,2022,50(01):102-110.
        \end{thebibliography}
\end{spacing}\vspace{0.5em} \section*{\heiti 附录} \vspace{0.5em}\begin{spacing}{1.25}\addcontentsline{toc}{section}{附录}
摘要又称概要、内容提要,意思是摘录要点或摘录下来的要点。 [1]  摘要是以提供文献内容梗概为目的,不加评论和补充解释,简明、确切地记述文献重要内容的短文。其基本要素包括研究目的、方法、结果和结论。具体地讲就是研究工作的主要对象和范围,采用的手段和方法,得出的结果和重要的结论,有时也包括具有情报价值的其它重要的信息。

摘要又称概要、内容提要,意思是摘录要点或摘录下来的要点。 [1]  摘要是以提供文献内容梗概为目的,不加评论和补充解释,简明、确切地记述文献重要内容的短文。其基本要素包括研究目的、方法、结果和结论。具体地讲就是研究工作的主要对象和范围,采用的手段和方法,得出的结果和重要的结论,有时也包括具有情报价值的其它重要的信息。

摘要又称概要、内容提要,意思是摘录要点或摘录下来的要点。 [1]  摘要是以提供文献内容梗概为目的,不加评论和补充解释,简明、确切地记述文献重要内容的短文。其基本要素包括研究目的、方法、结果和结论。具体地讲就是研究工作的主要对象和范围,采用的手段和方法,得出的结果和重要的结论,有时也包括具有情报价值的其它重要的信息。

摘要又称概要、内容提要,意思是摘录要点或摘录下来的要点。 [1]  摘要是以提供文献内容梗概为目的,不加评论和补充解释,简明、确切地记述文献重要内容的短文。其基本要素包括研究目的、方法、结果和结论。具体地讲就是研究工作的主要对象和范围,采用的手段和方法,得出的结果和重要的结论,有时也包括具有情报价值的其它重要的信息。
\end{spacing}\vspace{0.5em} \section*{\heiti 致谢} \vspace{0.5em}\begin{spacing}{1.25}\addcontentsline{toc}{section}{致谢}
摘要又称概要、内容提要,意思是摘录要点或摘录下来的要点。 [1]  摘要是以提供文献内容梗概为目的,不加评论和补充解释,简明、确切地记述文献重要内容的短文。其基本要素包括研究目的、方法、结果和结论。具体地讲就是研究工作的主要对象和范围,采用的手段和方法,得出的结果和重要的结论,有时也包括具有情报价值的其它重要的信息。
\end{spacing}\end{document}